
\documentclass[12pt]{article}
\usepackage{listings}
\usepackage{pdfpages}
\usepackage{amsmath}
\usepackage[legalpaper, margin=1in]{geometry}
\begin{document}
\begin{titlepage}
   \begin{center}
       \vspace*{1cm}
       \Large
       Exam 1
       \normalsize

       \vspace{0.5cm}

       Author: Gabriel Hofer

       \vspace{0.5cm}

       Course: CSC-410 Parallel Computing

       \vspace{0.5cm}

       Instructor: Dr. Karlsson
       \vspace{0.5cm}

       Due: October 19, 2020

       \vfill

       Computer Science and Engineering\

       South Dakota School of Mines and Technology\
   \end{center}
\end{titlepage}
%------------------------------------------------------------------------------------
\newpage
\subsection*{How to Make the Project}
\begin{lstlisting}[frame=single,language=Bash,caption=make \label{code:make}]
7356111@linux09 CSC-410-Exam-1 >>make
gcc openmp.c -fopenmp -lm -o openmp
nvcc cuda.cu -o cuda
pdflatex exam1.tex
...
\end{lstlisting}
\subsection*{Functions and Program Structure}
\textbf{main} calls either usage or range or correctness depending on the 
command line arguments. 

\textbf{Usage} prints a Message to standard output about how to start the program

\textbf{makeMatrix} makes a random matrix. the probability of that there is an edge 
for any two vertices is equal to $0.25$. We use the rand() C language
function to ``generate'' random integers. we also set the seed value 
before any rand() calls.

\textbf{serial} is our implementation of Floyd's algorithm without any parallelization.
We use it to check the correctness of our parallelized functions.

\textbf{Correctness}
The purpose of this function is to test whether our parallelized code 
is correct. We make the assumption that the function called serial (previously mentioned) 
is a correct implementation of Floyd's algorithm. So, we compare the 
output of our parallelized function to the output of serial. 

\textbf{Range} runs Floyd's algorithm in parallel for a range of values
of n. we iterate from small power of 2 to a greater power of 2. 

\textbf{printA} simply prints the 2D array, with a tab separating each column.

%--------------------------------------------------------------------
\newpage
\subsubsection*{Floyd CUDA}
\begin{lstlisting}[frame=single,language=C,caption=Floyd CUDA \label{code:make}]
__global__ void aux(int * dA, const int n, const int k){
  int index = threadIdx.x + blockIdx.x * blockDim.x;
  if(index >= n*n) return;
  __syncthreads();
  int i = index / n, j = index % n;
  dA[i*n+j] = dA[i*n+j] < (dA[i*n+k]+dA[k*n+j]) ? 
    dA[i*n+j] : dA[i*n+k]+dA[k*n+j];
  __syncthreads();
}

void floyd(int * dA, const int n){
  for(int k=0;k<n;k++){
    aux<<<(n*n+THREADS_PER_BLOCK)/(THREADS_PER_BLOCK),
      THREADS_PER_BLOCK>>>(dA,n,k);
    cudaDeviceSynchronize();
  }
}
\end{lstlisting}
\subsubsection*{Floyd OpenMP}
\begin{lstlisting}[frame=single,language=C,caption=Floyd OpenMP \label{code:make}]
void floyd(int * A, const int n){
  for(int k=0;k<n;k++)
    #pragma omp parallel for
    for(int i=0;i<n;i++)
      #pragma omp parallel for
      for(int j=0;j<n;j++)
        A[i*n+j] = A[i*n+j] < (A[i*n+k]+A[k*n+j]) ? 
          A[i*n+j] : A[i*n+k]+A[k*n+j];
}
\end{lstlisting}
In the OpenMP version of Floyd's algorithm, we insert two pragmas for the second and
third nested for loops.

%--------------------------------------------------------------------
\newpage
\subsection*{Testing and Verification}
\begin{lstlisting}[frame=single,language=C,caption=CUDA Correctness Testing \label{code:make}]
void correctness(const int low, const int high){
  for(int n = pow(2,low); n <= pow(2,high); n*=2){
    int * A = makeMatrix(n);
    int * B = (int *)malloc(n*n*sizeof(int));
    int Asize = n*n*sizeof(int);
    memcpy(B, A, Asize);
    serial(B,n);
    
    int * dA=NULL;
    cudaMalloc((void **)&dA, Asize);
    cudaMemcpy(dA, A, Asize, cudaMemcpyHostToDevice);
    floyd(dA,n);
    cudaMemcpy(A, dA, Asize, cudaMemcpyDeviceToHost);

    bool foundDiff=false;
    for(int i=0;i<n;i++)
      for(int j=0;j<n;j++)
        if(B[i*n+j]!=A[i*n+j]){
          foundDiff=true;
          return;
        }
    cudaFree(dA);
    free(A);
    free(B);
    cudaDeviceSynchronize();
    if(foundDiff){
      printf("FOUND DIFFERENCE:(\n\n");
      return;
    }
    printf("SAME\n");
  }
  printf("ALL SAME:)\n\n");
}
\end{lstlisting}
The correctness testing functions for CUDA and OpenMP are similar. 

%--------------------------------------------------------------------
\newpage
\subsection*{Usage Examples}
\begin{lstlisting}[frame=single,language=Bash,caption=CUDA Range Option \label{code:make}]
7356111@linux09 CSC-410-Exam-1 >>./cuda -r 1 12
2, 0.000023
4, 0.000031
8, 0.000059
16, 0.000113
32, 0.000224
64, 0.000483
128, 0.001060
256, 0.003196
512, 0.014511
1024, 0.113596
2048, 0.846598
4096, 6.665819
\end{lstlisting}
\begin{lstlisting}[frame=single,language=Bash,caption=OpenMP Range Option \label{code:make}]
7356111@linux09 CSC-410-Exam-1 >>./openmp -r 1 12
2, 0.000549
4, 0.000271
8, 0.000282
16, 0.000430
32, 0.001399
64, 0.001146
128, 0.008198
256, 0.033451
512, 0.168379
1024, 1.088660
2048, 7.835085
4096, 60.018689
\end{lstlisting}
The two listings above outputted 12 records with 2 fields in each record. The first field in each record is n and the second field in each record is the time it took to run Floyd's algorithm with an n by n matrix.



\subsection*{Charts and Analysis}

\subsection*{Deliverables}
\begin{enumerate}
	\item Makefile
	\item cuda.cu
	\item openmp.c
	\item exam1.pdf
\end{enumerate}

\end{document}

