
\documentclass[12pt]{article}
\usepackage{listings}
\usepackage{pdfpages}
\usepackage{amsmath}
\usepackage[legalpaper, margin=1in]{geometry}
\begin{document}
\begin{titlepage}
   \begin{center}
       \vspace*{1cm}
       \Large
       Exam 1
       \normalsize

       \vspace{0.5cm}

       Author: Gabriel Hofer

       \vspace{0.5cm}

       Course: CSC-410 Parallel Computing

       \vspace{0.5cm}

       Instructor: Dr. Karlsson
       \vspace{0.5cm}

       Due: October 19, 2020

       \vfill

       Computer Science and Engineering\

       South Dakota School of Mines and Technology\
   \end{center}
\end{titlepage}
%------------------------------------------------------------------------------------
\newpage
\subsection*{How to Make the Project}
\begin{verbatim}
7356111@linux09 CSC-410-Exam-1 >>make
gcc openmp.c -fopenmp -lm -o openmp
nvcc cuda.cu -o cuda
\end{verbatim}
\subsection*{Usage Statement}
\begin{verbatim}
Usage: ./cuda [-r low_power high_power] [-c low_power high_power]
	-r: runs Floyd's algorithm in parallel on range of powers of two
	-c: runs correctness tests on range of powers of two
\end{verbatim}
\subsection*{Usage Examples (Performance Tests for a Range of n)}
\begin{verbatim}
  7356111@linux09 CSC-410-Exam-1 >>./cuda -r 1 12
  2, 0.000023
  4, 0.000031
  8, 0.000059
  16, 0.000113
  32, 0.000224
  64, 0.000483
  128, 0.001060
  256, 0.003196
  512, 0.014511
  1024, 0.113596
  2048, 0.846598
  4096, 6.665819
\end{verbatim}
\begin{verbatim}
  7356111@linux09 CSC-410-Exam-1 >>./openmp -r 1 12
  2, 0.000549
  4, 0.000271
  8, 0.000282
  16, 0.000430
  32, 0.001399
  64, 0.001146
  128, 0.008198
  256, 0.033451
  512, 0.168379
  1024, 1.088660
  2048, 7.835085
  4096, 60.018689
\end{verbatim}
\subsection*{Data in a Table}
\begin{center}
  \begin{tabular}{ |c|c|c| } 
    \hline
    n & CUDA (sec) & OpenMP (sec) \\
    \hline
    2 &  0.000023 & 0.000549 \\
    4 & 0.000031 & 0.000271 \\
    8 & 0.000059 & 0.000282 \\
    16 & 0.000113 & 0.000430 \\
    32 & 0.000224 & 0.001399 \\
    64 & 0.000483 & 0.001146 \\
    128 & 0.001060 & 0.008198 \\
    256 & 0.003196 & 0.033451 \\
    512 & 0.014511 & 0.168379 \\
    1024 & 0.113596 & 1.088660 \\
    2048 & 0.846598 & 7.835085 \\
    4096 & 6.665819 & 60.018689 \\
    \hline
  \end{tabular}
\end{center}
\newpage
\subsection*{Usage Examples (Correctness Tests)}
\begin{verbatim}
7356111@linux09 CSC-410-Exam-1 >>./cuda -c 1 10
SAME
SAME
SAME
SAME
SAME
SAME
SAME
SAME
SAME
SAME
ALL SAME:)
\end{verbatim}
\begin{verbatim}
7356111@linux09 CSC-410-Exam-1 >>./openmp -c 1 10
SAME
SAME
SAME
SAME
SAME
SAME
SAME
SAME
SAME
SAME
ALL SAME:)
\end{verbatim}
For both options -r and -c, two integers are required to follow. These two integers are the range for $p$.
The two listings above show how to run cuda.cu and openmp.c from the command line using the -c option (which
runs correctness tests). Each output line that prints "SAME" is a successful test. 
And, "ALL SAME:)" means that all tests passed.  Each test is run on a different value of n calculated as follows:
\[
  n = 2^p 
\]
Where $p$ is a number between 1 and 10 which were the low\_power and high\_power arguments passed to the program.

\subsection*{Contents of exam1.tgz}
\begin{enumerate}
	\item Makefile
	\item cuda.cu
	\item openmp.c
	\item exam1.pdf
\end{enumerate}

%------------------------------------------------------------------------------
\newpage
\subsection*{Functions}
\begin{itemize}
  \item \textbf{main} calls either usage or range or correctness depending on the 
command line arguments. 
  \item \textbf{Usage} prints a Message to standard output about how to run the program.
  \item \textbf{makeMatrix} makes a random matrix. the probability of that there is an edge 
for any two vertices is equal to $0.25$. We use the rand() C language
function to ``generate'' random integers. we also set the seed value 
before any rand() calls.
  \item \textbf{serial} is our implementation of Floyd's algorithm without any parallelization.
We use it to check the correctness of our parallelized functions.
  \item \textbf{Correctness} tests whether our parallelized code 
is correct. We make the assumption that the function called serial 
is a correct implementation of Floyd's algorithm. So, we compare the 
output of our parallelized function to the output of serial. 
  \item \textbf{Range} runs Floyd's algorithm in parallel for a range of values
of n. we iterate from small power of 2 to a greater power of 2. 
  \item \textbf{printA} simply prints the 2D array, with a tab separating each column.
\end{itemize}
\subsection*{Files}
\begin{itemize}
  \item \textbf{cuda.cu} contains entire CUDA implementation of floyd's algorithm.
  \item \textbf{openmp.c} constains entire OpenMP implementation of floyd's algorithm.
\end{itemize}
%--------------------------------------------------------------------
\newpage
\subsubsection*{CUDA Floyd}
\begin{lstlisting}[frame=single,language=C,caption=CUDA Floyd \label{code:make}]
__global__ void aux(int * dA, const int n, const int k){
  int index = threadIdx.x + blockIdx.x * blockDim.x;
  if(index >= n*n) return;
  __syncthreads();
  int i = index / n, j = index % n;
  dA[i*n+j] = dA[i*n+j] < (dA[i*n+k]+dA[k*n+j]) ? 
    dA[i*n+j] : dA[i*n+k]+dA[k*n+j];
  __syncthreads();
}

void floyd(int * dA, const int n){
  for(int k=0;k<n;k++){
    aux<<<(n*n+THREADS_PER_BLOCK)/(THREADS_PER_BLOCK),
      THREADS_PER_BLOCK>>>(dA,n,k);
    cudaDeviceSynchronize();
  }
}
\end{lstlisting}
In CUDA, we use indexing to get rid of the two inner for loops in Floyd's algorithm.

\subsubsection*{OpenMP Floyd}
\begin{lstlisting}[frame=single,language=C,caption=OpenMP Floyd \label{code:make}]
void floyd(int * A, const int n){
  for(int k=0;k<n;k++)
    #pragma omp parallel for
    for(int i=0;i<n;i++)
      #pragma omp parallel for
      for(int j=0;j<n;j++)
        A[i*n+j] = A[i*n+j] < (A[i*n+k]+A[k*n+j]) ? 
          A[i*n+j] : A[i*n+k]+A[k*n+j];
}
\end{lstlisting}
In the OpenMP version of Floyd's algorithm, we insert two pragmas for the second and
third nested for loops.

\subsubsection*{Serial (Non-Parallelized)}
\begin{lstlisting}[frame=single,language=C,caption=Serial Floyd \label{code:make}]
void serial(int * A, const int n){
  for(int k=0;k<n;k++)
    for(int i=0;i<n;i++)
      for(int j=0;j<n;j++)
        A[i*n+j] = A[i*n+j] < (A[i*n+k]+A[k*n+j]) ? 
          A[i*n+j] : A[i*n+k]+A[k*n+j];
}
\end{lstlisting}

%--------------------------------------------------------------------
\newpage
\subsection*{Testing and Verification}
\begin{lstlisting}[frame=single,language=C,caption=CUDA Correctness Testing \label{code:make}]
void correctness(const int low, const int high){
  for(int n = pow(2,low); n <= pow(2,high); n*=2){
    int * A = makeMatrix(n);
    int * B = (int *)malloc(n*n*sizeof(int));
    int Asize = n*n*sizeof(int);
    memcpy(B, A, Asize);
    serial(B,n);
    
    int * dA=NULL;
    cudaMalloc((void **)&dA, Asize);
    cudaMemcpy(dA, A, Asize, cudaMemcpyHostToDevice);
    floyd(dA,n);
    cudaMemcpy(A, dA, Asize, cudaMemcpyDeviceToHost);

    bool foundDiff=false;
    for(int i=0;i<n;i++)
      for(int j=0;j<n;j++)
        if(B[i*n+j]!=A[i*n+j]){
          foundDiff=true;
          return;
        }
    cudaFree(dA);
    free(A);
    free(B);
    cudaDeviceSynchronize();
    if(foundDiff){
      printf("FOUND DIFFERENCE:(\n\n");
      return;
    }
    printf("SAME\n");
  }
  printf("ALL SAME:)\n\n");
}
\end{lstlisting}

\newpage
\subsection*{Experimental Serial Fraction}
\[
  e= \frac{\frac{1}{\psi}-\frac{1}{p}}{1-\frac{1}{p}}
\]
\end{document}

